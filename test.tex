\documentclass[english]{jamk-report}


\usepackage[utf8]{inputenc}
\usepackage[T1]{fontenc}
\usepackage[defaultsans]{opensans} %This font combo seems to be adequately hinted 
\usepackage{charter}
\usepackage[activate={true,nocompatibility},final,tracking=true,kerning=true,factor=1100,stretch=20,shrink=20]{microtype}
\usepackage{amsfonts,amsmath,amssymb,amsthm,booktabs,color,enumitem,graphicx}
\usepackage[pdftex,hidelinks]{hyperref}
\usepackage{datetime2} %ISO date format is recommended for language compatibility
\usepackage{acro}


% Automatically set the PDF metadata fields
\makeatletter
\AtBeginDocument{\hypersetup{pdftitle = {\@title}, pdfauthor = {\@author}}}
\makeatother

% babelbib for non-english bibliography using bibtex
\usepackage[fixlanguage]{babelbib}

% add bibliography to the table of contents
\usepackage[nottoc]{tocbibind}

% Document information

\title{JAMK \LaTeX\ Report Template}
\author{Samir Puuska, Marko Silokunnas}
\date{\today}
\level{Instruction manual}
\supervisors{Leslie Lamport}
\assigned{Liskomiehet}
\abstract{Jotai tuubaa tiivistelmäksi.}



% abbreviations:

\DeclareAcronym{tcp}{
  short = TCP ,
  long  = Transmission Control Protocol ,
  class = abbrev
}


\begin{document}

\frontmatter      % roman page numbering for front matter


\maketitle        % title page



\makeabstract     % abstract pages; If not writing thesis you may comment these out
\makeabstractfin

\tableofcontents  % table of contents

\listoffigures    %Remove if not needed
\listoftables

%Change these names to Finnish if needed
\addcontentsline{toc}{section}{List of Abbreviations}
\printacronyms[include-classes=abbrev,name=Abbreviations]

\mainmatter       % clear page, start arabic page numbering



\section{Introduction}

Latex\footnote{Pronounced not as latex the rubber!} is a powerful system
for creating professional looking documents, such as reports, thesis's, and
scientific papers.

This document strives to both help you in creating documents with \LaTeX, %https://oppimateriaalit.jamk.fi/raportointiohje/
as well as serve as a template that you can easily adjust to your needs.
This document follows the official JAMK report instructions as closely
as it is practical. Nevertheless, if your class or instructor requires
specific style, you may have to adjust this template. The font selection
is motivated by the on-screen readability factors; text hinting has
been confirmed to work in Adobe Acrobat and several Linux PDF readers.

The full template consists of three separate files: \texttt{template.tex}
is the main TeX file, where you produce your text. All document specific
options are adjustable in this file. \texttt{jamk-report.cls}
is the class file, that provides the basic report style. In most use cases there should be
no reason to adjust the template directly. Finally, \texttt{refs.bib} contains
an example bibliography in bibtex format. You should add your bibtex entries to this file.

While much work has been done to keep necessary adjustments minimal for
standard usage, there are still minor manual tweaks required if the user
wishes to change language. Some packages, such as the \texttt{acro} package,
do not support automatic localization. These exceptions are documented in the
\texttt{template.tex} file. This template currently supports Finnish and English.


\section{General Document Structure}

The first page of the JAMK report format is the title page. It has all the
elements one might expect to find on a title page, i.e. the title, author,
document type, and other descriptors. You should change these variables
in \texttt{template.tex} file.

\subsection{Class Options}

This template is based on the standard \texttt{article} class. The only option
that users should adjust is the language, by changing the documentclass entry
to desired option.

\begin{verbatim}
\documentclass[english]{jamk-report}
\end{verbatim}

Paper sizes other than A4 are not supported. It is not advisable to adjust text
size from the default 11p, although the package supports also 10p.

\subsection{Document Variables}

The template requires that the user sets certain variables, such as the title,
name of the author, and document type. In addition, if the user wishes to use
the \texttt{acro} environment for managing the abbreviations and acronyms, these
must also be declared at the document preamble.


\subsection{Sectioning and Structure}

The template uses standard convention for sectioning, i.e. 

\begin{verbatim}
\section{First Level}
\subsection{Second Level}
\subsubsection{Third Level}
\end{verbatim}

As expected, the table of contents is generated automatically.

\section{Tables and Images}

\section{Mathematics}

\section{Citations, Abbreviations, and Bibliography}

\section{crap}
\jamkfigure[2in]{images/test.png}{Test caption}{fig:dice}

Let's reference the test picture! Picture \ref{fig:dice} is a picture of
dice! Here's a test picture.\ac{tcp}



\begin{multline}\label{ccgpdf}
f(x)\equiv f(x;\mu_1,\mu_2,\sigma_1,\sigma_2,w_1,w_2,a,b)= \\
\left\{\begin{array}{ll}
0\;\;\; & \textrm{if}\; x<a \\
\frac{R}{\sqrt{2\pi}}\left[\frac {w_1}{\sigma_1}\exp\left(-\frac{(x-\mu_1)^2}{2\sigma_1^2}\right)+
\frac {w_2}{\sigma_2}\exp\left(-\frac{(x-\mu_2)^2}{2\sigma_2^2}\right)
\right]
\;\;\; & \textrm{if}\; a\le x\le b \\
0\;\;\; & \textrm{if}\; x>b \\
\end{array}\right.,
\end{multline}


Time for a table!

\jamktable
    {Different types of dice}   % Caption
    {tbl:dicetypes}             % Label that is used to refence the table
    {l r l}                     % Table layout
    {
        % Table data
        \textbf{Type} & \textbf{Number of sides} & \textbf{Usage} \\
        D4 & 4 & Tabletop RPGs \\
        D6 & 6 & Gambling, games... \\
        D10 & 10 & Tabletop RPGs \\
        D20 & 20 & Tabletop RPGs \\
        D100 & 100 & Tabletop RPGs \\
    }

Let's cite a reference without any additional information: 

Let's cite a reference some additional information (e.g. page number):

Suspendisse consequat lectus urna, vel hendrerit tortor euismod quis. Aliquam
diam urna, rhoncus at suscipit nec, pellentesque sed tellus. Fusce auctor
dignissim purus vel maximus. Morbi faucibus, lectus eget tempor posuere, tellus
ex imperdiet nibh, sed ultricies risus diam vitae nisi. Sed sed augue
malesuada, fringilla sem rutrum, molestie erat.  Phasellus non fringilla ex.
Integer eu lacinia est. Mauris commodo arcu eu consectetur condimentum. Donec
at ipsum at nisl blandit commodo a vitae nulla.  Quisque luctus sit amet turpis
vitae auctor. Nunc blandit metus ligula, nec auctor tortor sollicitudin et.
Nullam tristique efficitur ipsum, vel laoreet sapien lobortis at. Ut aliquet,
nulla id lobortis scelerisque, neque ante hendrerit turpis, vitae maximus sem
neque quis felis. Ut metus ante, sodales eu tincidunt eu, lobortis id mauris.

Let's reference our fine table: Table \ref{tbl:dicetypes} contains information
about different kinds of dice!~\cite{einstein}


\subsection{lorem ipsum}

Suspendisse consequat lectus urna, vel hendrerit tortor euismod quis. Aliquam
diam urna, rhoncus at suscipit nec, pellentesque sed tellus. Fusce auctor
dignissim purus vel maximus. Morbi faucibus, lectus eget tempor posuere, tellus
ex imperdiet nibh, sed ultricies risus diam vitae nisi. Sed sed augue
malesuada, fringilla sem rutrum, molestie erat.  Phasellus non fringilla ex.
Integer eu lacinia est. Mauris commodo arcu eu consectetur condimentum. Donec
at ipsum at nisl blandit commodo a vitae nulla.  Quisque luctus sit amet turpis
vitae auctor. Nunc blandit metus ligula, nec auctor tortor sollicitudin et.
Nullam tristique efficitur ipsum, vel laoreet sapien lobortis at. Ut aliquet,
nulla id lobortis scelerisque, neque ante hendrerit turpis, vitae maximus sem
neque quis felis. Ut metus ante, sodales eu tincidunt eu, lobortis id mauris.

\subsubsection{more lorem ipsum}

Suspendisse consequat facilisis lacus, eget varius neque. Mauris blandit id
tellus vel consectetur. Integer porta tempor arcu, quis sodales urna posuere a.
Phasellus a lacinia dolor. Nam nec dui massa. Praesent vestibulum purus ac
felis volutpat vehicula. Sed sem nisl, hendrerit id gravida at, condimentum
hendrerit massa.  Maecenas vitae erat laoreet, semper enim sit amet,
condimentum ipsum.~\cite{Puuska_2019,Haar_1910}

Donec at porttitor nibh. Suspendisse feugiat consequat ornare.  Mauris varius
porttitor libero ut facilisis. Pellentesque quis eros eros. Donec quis cursus
lorem. In eget diam felis. Sed dictum, tellus bibendum dictum commodo, ligula
felis semper nisl, ac molestie magna lacus vitae turpis. In volutpat nunc at
finibus vehicula. Vestibulum pretium at nibh in tempor. Cras sed mi sit amet
orci scelerisque mollis. Donec aliquet laoreet augue, ut malesuada massa semper
a. Suspendisse ac mi luctus, fringilla odio pellentesque, congue lorem.
Curabitur varius nunc eu elit mattis, sed gravida urna hendrerit. Duis eget
enim eget massa faucibus finibus. Suspendisse potenti. Interdum et malesuada
fames ac ante ipsum primis in faucibus.

Nam dapibus consequat sem, nec tempus nunc bibendum quis. Quisque vel felis
maximus, congue nibh vel, eleifend dolor. Duis accumsan orci et dolor pretium
tempor. Fusce vitae consequat magna. Nam ac risus lacus. Cras tortor ipsum,
cursus vel fermentum et, lobortis eget ante. Vestibulum consectetur porttitor
convallis. Mauris ultricies diam enim, sit amet auctor purus fermentum sed.
Fusce faucibus tincidunt mi, ac accumsan est iaculis a. Sed id lectus quam.
Maecenas bibendum, quam posuere faucibus posuere, dui mi finibus sapien, eu
efficitur odio dolor ac nibh. Etiam feugiat finibus viverra. Proin lobortis
porttitor arcu et malesuada.  Pellentesque rutrum venenatis leo, et luctus nibh
iaculis at.  Vestibulum ac nulla quis tellus rutrum laoreet ac sit amet lacus.

\bibliographystyle{acm} %ACM is ok for engineering and math. Change if needed.
\bibliography{refs}

\end{document}
