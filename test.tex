\documentclass[english]{jamk-report}


\usepackage[utf8]{inputenc}
\usepackage[T1]{fontenc}

\usepackage{charter}
\linespread{1.05} %A bit more air for charter
\usepackage[charter]{mathdesign}

\usepackage[activate={true,nocompatibility},final,tracking=true,kerning=true,factor=1100,stretch=20,shrink=20]{microtype}
\microtypesetup{expansion=false}

\usepackage{booktabs,color,enumitem,tabularx}
\usepackage[pdftex,hidelinks]{hyperref}
\usepackage{datetime2} %ISO date format is recommended for language compatibility

% Automatically set the PDF metadata fields
\makeatletter
\AtBeginDocument{\hypersetup{pdftitle = {\@title}, pdfauthor = {\@author}}}
\makeatother

% babelbib for non-english bibliography using bibtex
\usepackage[fixlanguage]{babelbib}

% add bibliography to the table of contents
\usepackage[nottoc]{tocbibind}

% Document information

\title{JAMK \LaTeX\ Report Template}
\author{Samir Puuska, Marko Silokunnas}
\date{\today}
\level{Instruction manual}
\supervisors{Leslie Lamport}
\assigned{Liskomiehet}
\abstract{Jotai tuubaa tiivistelmäksi.}



% abbreviations:

\DeclareAcronym{tcp}{
  short = TCP ,
  long  = Transmission Control Protocol ,
  class = abbrev
}


\begin{document}

\frontmatter      % roman page numbering for front matter


\maketitle        % title page



\makeabstract     % abstract pages; If not writing thesis you may comment these out
\makeabstractfin

\tableofcontents  % table of contents

\listoffigures    %Remove if not needed
\listoftables

%Change these names to Finnish if needed
\addcontentsline{toc}{section}{List of Abbreviations}
\printacronyms[include-classes=abbrev,name=Abbreviations]

\mainmatter       % clear page, start arabic page numbering



\section{Introduction}

Latex\footnote{Pronounced not as latex the rubber!} is a powerful system
for creating professional looking documents, such as reports, thesis's, and
scientific papers.

This document strives to both help you in creating documents with \LaTeX, %https://oppimateriaalit.jamk.fi/raportointiohje/
as well as serve as a template that you can easily adjust to your needs.
This document follows the official JAMK report instructions as closely
as it is practical. Nevertheless, if your class or instructor requires
specific style, you may have to adjust this template. The font selection
is motivated by the on-screen readability factors; text hinting has
been confirmed to work in Adobe Acrobat and several Linux PDF readers.

The full template consists of three separate files: \texttt{template.tex}
is the main TeX file, where you produce your text. All document specific
options are adjustable in this file. \texttt{jamk-report.cls}
is the class file, that provides the basic report style. In most use cases there should be
no reason to adjust the template directly. Finally, \texttt{refs.bib} contains
an example bibliography in bibtex format. You should add your bibtex entries to this file.

While much work has been done to keep necessary adjustments minimal for
standard usage, there are still minor manual tweaks required if the user
wishes to change language. Some packages, such as the \texttt{acro} package,
do not support automatic localization. These exceptions are documented in the
\texttt{template.tex} file. This template currently supports Finnish and English.


\section{General Document Structure}

The first page of the JAMK report format is the title page. It has all the
elements one might expect to find on a title page, i.e. the title, author,
document type, and other descriptors. You should change these variables
in \texttt{template.tex} file.

\subsection{Class Options}

This template is based on the standard \texttt{article} class. The only option
that users should adjust is the language, by changing the documentclass entry
to desired option.

\begin{verbatim}
\documentclass[english]{jamk-report}
\end{verbatim}

\noindent
Paper sizes other than A4 are not supported. It is not advisable to adjust text
size from the default 11p, although the package supports also 10p.

\subsection{Document Variables}

The template requires that the user sets certain variables, such as the title,
name of the author, and document type. In addition, if the user wishes to use
the \texttt{acro} environment for managing the abbreviations and acronyms, these
must also be declared at the document preamble. Table~\ref{table:vars} lists
the variables and their use.



\begin{table}[h]\centering
\ra{1.3}
    \begin{tabularx}{\textwidth-1em}{p{6em}l>{\raggedleft\arraybackslash}X} \toprule
        Variable & Purpose & Mandatory  \\ \midrule
        \texttt{title} & The main title & Yes \\
        \texttt{author} & The author(s) & Yes \\
        \texttt{date} & Date of e.g. acceptance & Yes \\
        \texttt{level} & Document type, e.g. thesis or report & Yes \\
        \texttt{abstract} & The abstract & When description pages enabled \\
        \texttt{supervisors} & Supervisors of e.g. for thesis & No \\
        \texttt{assigned} & If assigned by a third party & No \\

        \bottomrule
    \end{tabularx}
\caption{List of variables that describe different aspects of the document.}
\label{table:vars}
\end{table}


\subsection{Sectioning and Structure}

The template uses standard convention for sectioning, i.e. 

\begin{verbatim}
\section{First Level}
\subsection{Second Level}
\subsubsection{Third Level}
\end{verbatim}

\noindent
As expected, the table of contents is generated automatically. This template creates
the list of figures, tables, and abbreviations by default. If you wish not to include
these sections, you can comment them out in the latex source.

\section{Tables and Images}


This section covers the basics of using both tables and images. It is likely
that you may need to adjust or alter the default parameters to suit your
specific need. In general, you should use search
engines to find what solutions others have used. It is likely that
what you are attempting to do has already been done.

\subsection{Tables}

Tables are an art form. It is generally advisable to omit as much lines
as possible while maintaining clarity. Table~\ref{table:web} is an example,
where spacing is used to separate the items. 

\begin{table}[h]\centering
\ra{1.3}
    \begin{tabularx}{\textwidth-1em}{llc} \toprule
        Abbreviation & Technology & Organization  \\ \midrule
        HTML 5.1 & Hypertext Markup Language & W3C \\
        CSS  & Cascading Style Sheets & W3C  \\
        HTTP/1.1& Hypertext Transfer Protocol &  IETF \\
        TLS 1.1 & Transport Layer Security & IETF \\
        JSON & JavaScript Object Notation & IETF \\
        ES & ECMAScript & Ecma International \\
        \bottomrule
    \end{tabularx}
\caption{Web standards and governing standardization organizations.}
\label{table:web}
\end{table}

\noindent
In many cases the example table is not adequate for your data. The table
may prove to be too narrow for large datasets. In cases like these you
should consider moving the table to a separate page, and changing the
orientation to landscape. For this you can try e.g. \texttt{sidewaystable} 
from the \texttt{rotating} package. There is also a package, \texttt{pdflscape}, that automatically
orients a single PDF page to landscape mode in PDF readers.


\subsection{Images}

Latex supports almost every common image and vector format with appropriate 
packages. Vector formats such as \texttt{.eps} or \texttt{.pdf} (with an embedded vector image)
are highly recommended.

For pictures that can not be vectorized, \texttt{.png} format is recommended. You
may also use \texttt{.jpeg} for photographs. Note, that if you require accurate
reproduction of colors in print, you may need to use CMYK colors, and worry about different gamuts.

\section{Mathematics}

\section{Citations, Abbreviations, and Bibliography}

\section{crap}
\jamkfigure[2in]{images/test.png}{Test caption}{fig:dice}

Let's reference the test picture! Picture \ref{fig:dice} is a picture of
dice! Here's a test picture.\ac{tcp}



\begin{eqnarray}
    u_\alpha & = & \epsilon^2 \kappa_{xxx} 
    \left( y-\frac{1}{2}y^2 \right),
    \label{equ}  \\
    v & = & \epsilon^3 \kappa_{xxx} y\,,
    \label{eqv}  \\
    p & = & \epsilon \kappa_{xx}\,.
    \label{eqp}
\end{eqnarray}

\begin{eqnarray}
    \omega_1 & = &
    \frac{\partial w}{\partial y}-\frac{\partial v}{\partial z}\,,
    \nonumber  \\
    \omega_2 & = & 
    \frac{\partial u}{\partial z}-\frac{\partial w}{\partial x}\,,
    \label{eqcurl}  \\
    \omega_3 & = & 
    \frac{\partial v}{\partial x}-\frac{\partial u}{\partial y}\,.
    \nonumber
\end{eqnarray}





Let's cite a reference without any additional information: 

Let's cite a reference some additional information (e.g. page number):

Suspendisse consequat lectus urna, vel hendrerit tortor euismod quis. Aliquam
diam urna, rhoncus at suscipit nec, pellentesque sed tellus. Fusce auctor
dignissim purus vel maximus. Morbi faucibus, lectus eget tempor posuere, tellus
ex imperdiet nibh, sed ultricies risus diam vitae nisi. Sed sed augue
malesuada, fringilla sem rutrum, molestie erat.  Phasellus non fringilla ex.
Integer eu lacinia est. Mauris commodo arcu eu consectetur condimentum. Donec
at ipsum at nisl blandit commodo a vitae nulla.  Quisque luctus sit amet turpis
vitae auctor. Nunc blandit metus ligula, nec auctor tortor sollicitudin et.
Nullam tristique efficitur ipsum, vel laoreet sapien lobortis at. Ut aliquet,
nulla id lobortis scelerisque, neque ante hendrerit turpis, vitae maximus sem
neque quis felis. Ut metus ante, sodales eu tincidunt eu, lobortis id mauris.

Let's reference our fine table: Table \ref{tbl:dicetypes} contains information
about different kinds of dice!~\cite{einstein}


\subsection{lorem ipsum}

Suspendisse consequat lectus urna, vel hendrerit tortor euismod quis. Aliquam
diam urna, rhoncus at suscipit nec, pellentesque sed tellus. Fusce auctor
dignissim purus vel maximus. Morbi faucibus, lectus eget tempor posuere, tellus
ex imperdiet nibh, sed ultricies risus diam vitae nisi. Sed sed augue
malesuada, fringilla sem rutrum, molestie erat.  Phasellus non fringilla ex.
Integer eu lacinia est. Mauris commodo arcu eu consectetur condimentum. Donec
at ipsum at nisl blandit commodo a vitae nulla.  Quisque luctus sit amet turpis
vitae auctor. Nunc blandit metus ligula, nec auctor tortor sollicitudin et.
Nullam tristique efficitur ipsum, vel laoreet sapien lobortis at. Ut aliquet,
nulla id lobortis scelerisque, neque ante hendrerit turpis, vitae maximus sem
neque quis felis. Ut metus ante, sodales eu tincidunt eu, lobortis id mauris.

\subsubsection{more lorem ipsum}

Suspendisse consequat facilisis lacus, eget varius neque. Mauris blandit id
tellus vel consectetur. Integer porta tempor arcu, quis sodales urna posuere a.
Phasellus a lacinia dolor. Nam nec dui massa. Praesent vestibulum purus ac
felis volutpat vehicula. Sed sem nisl, hendrerit id gravida at, condimentum
hendrerit massa.  Maecenas vitae erat laoreet, semper enim sit amet,
condimentum ipsum.~\cite{Puuska_2019,Haar_1910}

Donec at porttitor nibh. Suspendisse feugiat consequat ornare.  Mauris varius
porttitor libero ut facilisis. Pellentesque quis eros eros. Donec quis cursus
lorem. In eget diam felis. Sed dictum, tellus bibendum dictum commodo, ligula
felis semper nisl, ac molestie magna lacus vitae turpis. In volutpat nunc at
finibus vehicula. Vestibulum pretium at nibh in tempor. Cras sed mi sit amet
orci scelerisque mollis. Donec aliquet laoreet augue, ut malesuada massa semper
a. Suspendisse ac mi luctus, fringilla odio pellentesque, congue lorem.
Curabitur varius nunc eu elit mattis, sed gravida urna hendrerit. Duis eget
enim eget massa faucibus finibus. Suspendisse potenti. Interdum et malesuada
fames ac ante ipsum primis in faucibus.

Nam dapibus consequat sem, nec tempus nunc bibendum quis. Quisque vel felis
maximus, congue nibh vel, eleifend dolor. Duis accumsan orci et dolor pretium
tempor. Fusce vitae consequat magna. Nam ac risus lacus. Cras tortor ipsum,
cursus vel fermentum et, lobortis eget ante. Vestibulum consectetur porttitor
convallis. Mauris ultricies diam enim, sit amet auctor purus fermentum sed.
Fusce faucibus tincidunt mi, ac accumsan est iaculis a. Sed id lectus quam.
Maecenas bibendum, quam posuere faucibus posuere, dui mi finibus sapien, eu
efficitur odio dolor ac nibh. Etiam feugiat finibus viverra. Proin lobortis
porttitor arcu et malesuada.  Pellentesque rutrum venenatis leo, et luctus nibh
iaculis at.  Vestibulum ac nulla quis tellus rutrum laoreet ac sit amet lacus.

\bibliographystyle{acm} %ACM is ok for engineering and math. Change if needed.
\bibliography{refs}

\end{document}
